\documentclass[preprint,12pt]{article}
\usepackage{url}
\usepackage{cite}
\usepackage{epsfig}
\usepackage{hyperref}

\begin{document}

\title{2018 Proposal Flux Plots}
\author{Brian Clark}

\maketitle

\section{Flux Estimates and Measurements}

\subsection{Theoretical Predictions}
Two very common theoretical predictions for the GZK cosmogenic neutrino flux are those by \textit{Ahlers and Halzen} \cite{Ahlers:2012rz} and by \textit{Kotera et. al.} \cite{Kotera:2010yn}.

\subsection{IceCube Measurement}


We will utilize three of the most recent IceCube flux measurements. The first is their measurement of the astrophysical muon neutrino spectrum using eight-years of through-going muons \cite{PoS(ICRC2017)1005}:
\begin{equation}
\frac{d \Phi _ {\nu + \bar{\nu}}}{d E} = 3 \times (1.01^{+0.26}_{-0.23}) \Bigg( \frac{E}{100~\textrm{TeV} } \Bigg) ^{-2.19 \pm 0.10} \cdot 10^{-18} ~\textrm{GeV} ~\textrm{cm}^{-2} ~\textrm{s}^{-1} ~\textrm{sr}^{-1}
\end{equation}

The second is their measurement of the all-flavor astrophysical neutrino spectrum using four years of cascades \cite{PoS(ICRC2017)968}:
\begin{equation}
\frac{d \Phi _ {\nu + \bar{\nu}}}{d E} = 3 \times (1.57^{+0.23}_{-0.22}) \Bigg( \frac{E}{100~\textrm{TeV} } \Bigg) ^{-2.48 \pm 0.08} \cdot 10^{-18} ~\textrm{GeV} ~\textrm{cm}^{-2} ~\textrm{s}^{-1} ~\textrm{sr}^{-1}
\end{equation}

Their most recent \textit{peer-reviewed} result is their combined-likelihood analysis, which utilizes both tracks and cascades \cite{Aartsen:2015knd}:
\begin{equation}
\frac{d \Phi _ {\nu + \bar{\nu}}}{d E} = (6.7^{+1.1}_{-1.2}) \Bigg( \frac{E}{100~\textrm{TeV} } \Bigg) ^{-2.50 \pm 0.09} \cdot 10^{-18} ~\textrm{GeV} ~\textrm{cm}^{-2} ~\textrm{s}^{-1} ~\textrm{sr}^{-1}
\end{equation}

It is interesting to note that the spectral index $\gamma$ in the muon-based measurement is considerably harder than the explicitly all-flavor measurements, but the tension is only $\sim 2\sigma$.

\section{Event Number Estimate}
To estimate the number of events that would be detected by an experiment, we must complete the following integral:
\begin{equation}
N = \int \bigg( \frac{dN}{dE dA d\Omega dt} \bigg) [ \Omega A_{{eff}}] ~dt ~dE
\end{equation}
where we are integrating over a flux model. This can be discretized into a sum over energy bins:
\begin{equation}
N = \Delta t ~ \sum_i \bigg( \frac{dN}{dE dA d\Omega dt} \bigg)_i [ \Omega A_{eff}]_i ~\Delta E_i
\end{equation}



%\begin{figure}[h]
%\centering
%\includegraphics[width=.9\textwidth]{leading_limits_plot.pdf}
%\caption{Some words.}
%\label{fig:sigcondblock1}
%\end{figure}


\bibliographystyle{elsarticle-num}
%\bibliographystyle{unsrt}
\bibliography{references} 

\end{document}
